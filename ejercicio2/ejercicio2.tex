\ifunoporpagina
\newpage
\fi
\section*{Ejercicio numero 2}
%aca va el ejercicio

\begin{table}[]
\begin{tabular}{|l|l|l|l|l|l|l|}
\hline
                                                                            & Vcc                                                        & Voltage (25 C)                                              & Vcc         & Voltage (25 C) & Vcc      & Voltage (25 C)   \\ \hline
74HC02                                                                      & \multicolumn{2}{l|}{74HC02}                                                                                              & \multicolumn{2}{l|}{74HCT02} & \multicolumn{2}{l|}{74LS02} \\ \hline
\begin{tabular}[c]{@{}l@{}}Minimum HIGH Level\\ Input Voltage\end{tabular}  & \begin{tabular}[c]{@{}l@{}}2.0V\\ 4.5V\\ 6.0V\end{tabular} & \begin{tabular}[c]{@{}l@{}}1.5V\\ 3.15V\\ 4.2V\end{tabular} & 4.5V a 5.5V & 2V             & 4.75V    & 2V               \\ \hline
\begin{tabular}[c]{@{}l@{}}Maximum LOW Level\\ Input Voltage\end{tabular}   & \begin{tabular}[c]{@{}l@{}}2.0V\\ 4.5V\\ 6.0V\end{tabular} & \begin{tabular}[c]{@{}l@{}}0.5V\\ 1.35V\\ 1.8V\end{tabular} & 4.5V a 5.5V & 0.8V           & 4.75V    & 0.8V             \\ \hline
\begin{tabular}[c]{@{}l@{}}Minimum HIGH Level\\ Output Voltage\end{tabular} & \begin{tabular}[c]{@{}l@{}}2.0V\\ 4.5V\\ 6.0V\end{tabular} & \begin{tabular}[c]{@{}l@{}}1.9V\\ 4.4V\\ 5.9V\end{tabular}  & 4.5V a 5.5V & 4.4V           & 4.75V    & 2.7V             \\ \hline
\begin{tabular}[c]{@{}l@{}}Maximum LOW Level\\ Output Voltage\end{tabular}  & \begin{tabular}[c]{@{}l@{}}2.0V\\ 4.5V\\ 6.0V\end{tabular} & \begin{tabular}[c]{@{}l@{}}0.1V\\ 0.1V\\ 0.1V\end{tabular}  & 4.5V a 5.5V & 0.1V           & 4.75V    & 0.5V             \\ \hline
\end{tabular}
\end{table}

\iffalse
Seccion de cosas utiles para agregar

texto----------------------------
\textit[texto en negrita]




listas --------------------------------------------------
\begin{list_type}  
\item The first item 
\item The second item 
\item The third etc \ldots 
\end{list_type}

list_type es:
itemize for a bullet list
enumerate for an enumerated list and
description for a descriptive list.

ejemplo listas anidadas-----------------------------------------------

\begin{enumerate}
\item The first item
\begin{enumerate}
\item Nested item 1
\item Nested item 2
\end{enumerate}
\item The second item
\item The third etc \ldots
\end{enumerate}

tildes-------------------------------
babel ya esta incluido, si hay que poner tildes se ponen \acute{i}
ejemplo: l\acute{i}mite

tip: escriban normal y despues hagan replace

subsecciones-------------------------------

si necesitan subsecciones le clavan un buen \subsection*{nombre}

y si estan en piolas y necesitan sub sub secciones le clavan un \subsubsection*{nombre}

quien hubiera dicho

figuras----------------------------------------

\begin{figure}% este es para caption arriba o abajo
  \caption{A picture of a gull.} %caption arriba
  \centering
    \includegraphics[width=0.5\textwidth]{gull}
    \caption{A picture of a gull.} %caption abajo
\end{figure}

\begin{SCfigure} %este es para caption al costado
  \centering
  \caption{ ... caption text ... } 
  \includegraphics[width=0.3\textwidth]%
    {Giraffe_picture}% picture filename
\end{SCfigure}

para hacer que la figure se quede quietecita:

\begin{figure}[letrita_placement]

donde letrita_placement es:
h	Place the float here, i.e., approximately at the same point it occurs in the source text (however, not exactly at the spot)
t	Position at the top of the page.
b	Position at the bottom of the page.
p	Put on a special page for floats only.
!	Override internal parameters LaTeX uses for determining "good" float positions.
H	Places the float at precisely the location in the LaTeX code. Requires the float package,[1] i.e., \usepackage{float}. This is somewhat equivalent to !ht.

\fi