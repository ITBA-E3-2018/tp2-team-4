\begin{circuitikz}[scale=1]
\draw(-1.5,-2) rectangle (1.5,2); %IC rectangle
\draw (0,0.5) node [align=center]{\large xx-555\\TIMER}; % IC LABEL
% Draw the pins

\draw (0.9,-2) node [above]{GND} -- +(0,-0.5) node [anchor=-45]{1} coordinate (GND); % Pin 1 GND
\draw (-1.5,-1.5) node [right]{TRG} -- +(-0.5,0) node [anchor=-135]{2} coordinate (TRG); % Pin 2 TRG
\draw (1.5,0) node [left]{OUT} -- +(0.5,0) node [anchor=-45]{3} coordinate (OUT); % Pin 3 OUT  
\draw (0.9,2) node [below]{RESET} -- +(0,0.5) node [anchor=45]{4} coordinate (RESET); % Pin 4 RESET
\draw (0,-2) node [above]{CTRL} -- +(0,-0.5) node [anchor=-45]{5} coordinate (CTRL); % Pin 5 CTRL
\draw (-1.5,-.5) node [right]{THR} -- +(-0.5,0) node [anchor=-135]{6} coordinate (THR); % Pin 6 THR
\draw (-1.5,1.5) node [right]{DIS} -- +(-0.5,0) node [anchor=-135]{7} coordinate (DIS); % Pin 7 DIS
\draw (0,2) node [below]{$\mathsf{V_{CC}}$} -- +(0,0.5) node [anchor=45]{8} coordinate (VCC); % Pin 8 VCC

% Start conecting 
\draw(-4,3.5) % Start point
    node [anchor=east]{$V_{CC}$}
    to [short, o-] ++(1,0) coordinate (NOD1) % Use auxiliar coordinate (NOD1)
    to [R=$R_1$,*-*] (DIS -| NOD1) node(NOD3){}% to the point in the intersection between NOD1 and 1 DIS
    to [R=$R_2$,*-*] (THR -| NOD1) node(NOD2){}% idem
    to [short, *-*] (TRG -| NOD1)
    to [C=C,*-*] (-3,-5)
    to [short,*-o] ++(-1,0) coordinate (GND2)
    node [anchor=east]{GND};
    
    %Conect U-1
\draw (VCC) to [short, -*] (VCC |- NOD1);
\draw (RESET) to [short, -*] (RESET |- NOD1);
\draw (TRG) to [short, -*] (TRG -| NOD1);
\draw (CTRL) to [C,l_=0.01nF, -*] (CTRL |- GND2);
\draw (GND) to [short, -*] (GND |- GND2);
\draw (GND2) to [short] (GND2-|GND);
\draw (NOD2) to [short] (THR);
\draw (NOD3) to[short] (DIS);
\draw (NOD1) to[short] (NOD1-|RESET);
\draw (OUT) to[short,-o] ++(right:1) node[right]{Clk \texttiming{LHLHL}};
\end{circuitikz}