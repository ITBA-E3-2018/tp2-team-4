\part*{Ejercicio 2}
Procedimos a alimentar una compuerta del integrado LS02 con la salida de una misma compuerta pero del integrado HC02, y luego alimentamos de la misma manera pero en sentido inverso. Pudimos notar que hay zonas donde el circuito armado no deber\'ia andar de forma \'optima por el margen de ruido que manejan las distintas compuertas pero funciona igual ya que la ca\'ida de tensi\'on en el LS02 no es tan grande y alcanza a caer cerca del l\'imite del HC02 con 4,2V aproximadamente que es el m\'inimo del estado HIGH para el HC02. Si hubiera sido menor el valor de la tensi\'on no hubi\'eramos obtenido alguna salida por lo marcado en las hojas de datos.
\newline
El \textbf{fan-out} est\'a determinado por la cantidad de corriente que puede aceptar en la entrada cada CI y es la cantidad de pines que puede alimentar un CI con alguna de sus salidas. En la tecnolog\'ia CMOS(HC) seg\'un su hoja de dato acepta 20mA como m\'aximo, mientras que la tecnolog\'ia TTL(LS) acepta 0,4mA como m\'aximo en la entrada y 8mA en la salida. Haciendo las cuentas directas de estos casos, con la salida de un integrado LS puedo alimentar hasta 20 entradas LS, mientras que con un HC puedo alimentar 50 entradas LS. Cabe destacar que en la hoja de datos que brinda el fabricante solo asegura el funcionamiento de hasta 10 pines LS-TTL con una salida del HC, el cual debe ser para el peor caso que puede surgir para este integrado.
\newline
Al hacer las mediciones alimentando con el HCT y notamos un comportamiento mejor en cuanto la alimentaci\'on del LS02, ya que la tecnolog\'ia HCT es a base de CMOS pero tiene una gran tolerancia con la tecnolog\'ia TTL en cuanto a los valores de tensi\'on.
