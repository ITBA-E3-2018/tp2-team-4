

%aca va el ejercicio

Se implementaron dos compuertas NOT con tecnolog�a BJT: una variante con un transistor NPN () y otra con un transistor PNP (). Sus dise�os se pueden observar en las figuras.%insertar referencias a figuras de dise�o
Midiendo con un osciloscopio la entrada y la salida del circuito y haciendo uso del modo XY, se obtuvo la curva caracter\'istica de tensi\'on de cada compuerta, que se observan en las figuras.%insertar referencias a figuras de dise�o
A partir de estas curvas, se obtuvieron los niveles de voltaje de input y output para los niveles altos y bajos de ambas compuertas, as\'i como los m\'argenes de ruido. Estos figuran en la tabla.%insertar referencia a tabla de niveles de tension
Posteriormente, midiendo las curvas de entrada y salida en simult\'aneo, se obtuvieron los tiempos de transici\'on y las demoras de propagaci\'on para ambas compuertas. Estos figuran en la tabla.%insertar referencia a tabla de tiempos
Luego, se cargaron las compuertas con un capacitor, y obteniendo la derivada de la tensi�n de salida pudo determinarse la corriente m\'axima a trav\'es de la compuerta.

 

\iffalse
Seccion de cosas utiles para agregar

texto----------------------------
\textit[texto en negrita]




listas --------------------------------------------------
\begin{list_type}  
\item The first item 
\item The second item 
\item The third etc \ldots 
\end{list_type}

list_type es:
itemize for a bullet list
enumerate for an enumerated list and
description for a descriptive list.

ejemplo listas anidadas-----------------------------------------------

\begin{enumerate}
\item The first item
\begin{enumerate}
\item Nested item 1
\item Nested item 2
\end{enumerate}
\item The second item
\item The third etc \ldots
\end{enumerate}

tildes-------------------------------
babel ya esta incluido, si hay que poner tildes se ponen \acute{i}
ejemplo: l\acute{i}mite

tip: escriban normal y despues hagan replace

subsecciones-------------------------------

si necesitan subsecciones le clavan un buen \subsection*{nombre}

y si estan en piolas y necesitan sub sub secciones le clavan un \subsubsection*{nombre}

quien hubiera dicho

figuras----------------------------------------

\begin{figure}% este es para caption arriba o abajo
  \caption{A picture of a gull.} %caption arriba
  \centering
    \includegraphics[width=0.5\textwidth]{gull}
    \caption{A picture of a gull.} %caption abajo
\end{figure}

\begin{SCfigure} %este es para caption al costado
  \centering
  \caption{ ... caption text ... } 
  \includegraphics[width=0.3\textwidth]%
    {Giraffe_picture}% picture filename
\end{SCfigure}

para hacer que la figure se quede quietecita:

\begin{figure}[letrita_placement]

donde letrita_placement es:
h	Place the float here, i.e., approximately at the same point it occurs in the source text (however, not exactly at the spot)
t	Position at the top of the page.
b	Position at the bottom of the page.
p	Put on a special page for floats only.
!	Override internal parameters LaTeX uses for determining "good" float positions.
H	Places the float at precisely the location in the LaTeX code. Requires the float package,[1] i.e., \usepackage{float}. This is somewhat equivalent to !ht.

\fi